\newpage
\section{面试题59 - II. 队列的最大值}
\label{leetcode:sw_59_2}

\subsection{题目}

请定义一个队列并实现函数 max\_value 得到队列里的最大值,
要求函数 max\_value、push\_back 和 pop\_front 的时间复杂度都是O(1)。

若队列为空,pop\_front 和 max\_value 需要返回 -1

\textbf{示例 1}:

\begin{verbatim}
  输入:
  ["MaxQueue","push_back","push_back","max_value","pop_front","max_value"]
  [[],[1],[2],[],[],[]]
  输出: [null,null,null,2,1,2]
\end{verbatim}

\textbf{示例 2}:

\begin{verbatim}
  输入:
  ["MaxQueue","pop_front","max_value"]
  [[],[],[]]
  输出: [null,-1,-1]
\end{verbatim}

\textbf{限制}:

\begin{verbatim}
  1 <= push_back,pop_front,max_value的总操作数 <= 10000
  1 <= value <= 10^5
\end{verbatim}

\subsection{参考题解}

\subsubsection{Python}

\begin{verbatim}
class MaxQueue:

  def __init__(self):
    self.maxqueue = []
    self.queue = []

  def max_value(self) -> int:
    if len(self.maxqueue) == 0: return -1
    return self.maxqueue[0]

  def push_back(self, value: int) -> None:
    self.queue.append(value)
    for i in range(len(self.maxqueue) - 1, -1, -1):
      if self.maxqueue[i] < value:
        self.maxqueue.pop()
    self.maxqueue.append(value)

  def pop_front(self) -> int:
    if len(self.queue) == 0: return -1
    value = self.queue.pop(0)
    if value == self.maxqueue[0]:
      self.maxqueue.pop(0)
    return value

# Your MaxQueue object will be instantiated and called as such:
# obj = MaxQueue()
# param_1 = obj.max_value()
# obj.push_back(value)
# param_3 = obj.pop_front()
\end{verbatim}

\subsection{相似题目}

\hyperref[leetcode:239]{239. 滑动窗口最大值}
