\newpage
\section{438. 找到字符串中所有字母异位词}
\label{leetcode:438}

\subsection{题目}

给定一个字符串 s 和一个非空字符串 p,找到 s 中所有是 p 的字母异位词的子串,返回这些子串的起始索引。

字符串只包含小写英文字母,并且字符串 s 和 p 的长度都不超过 20100。

\textbf{说明}:

\begin{itemize}
  \item 字母异位词指字母相同,但排列不同的字符串。
  \item 不考虑答案输出的顺序。
\end{itemize}

\textbf{示例 1}:

\begin{verbatim}
  输入:
  s: "cbaebabacd" p: "abc"

  输出:
  [0, 6]

  解释:
  起始索引等于 0 的子串是 "cba", 它是 "abc" 的字母异位词。
  起始索引等于 6 的子串是 "bac", 它是 "abc" 的字母异位词。
\end{verbatim}

\textbf{示例 2}:

\begin{verbatim}
  输入:
  s: "abab" p: "ab"

  输出:
  [0, 1, 2]

  解释:
  起始索引等于 0 的子串是 "ab", 它是 "ab" 的字母异位词。
  起始索引等于 1 的子串是 "ba", 它是 "ab" 的字母异位词。
  起始索引等于 2 的子串是 "ab", 它是 "ab" 的字母异位词。
\end{verbatim}

\subsection{参考题解,滑动窗口}

\subsubsection{Python}

\begin{verbatim}
class Solution:
  def findAnagrams(self, s: str, p: str) -> List[int]:
    result = []
    pCounter = collections.Counter(p)
    sCounter = collections.Counter(s[:len(p)])
    for i in range(len(s) - len(p) + 1):
      if sCounter == pCounter:
        result.append(i)
      if sCounter[s[i]] > 1:
        sCounter[s[i]] -= 1
      else:
        del sCounter[s[i]]
      if i + len(p) < len(s):
        sCounter[s[i + len(p)]] += 1
    return result
\end{verbatim}
