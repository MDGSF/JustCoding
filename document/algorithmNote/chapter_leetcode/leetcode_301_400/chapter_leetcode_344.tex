\newpage
\section{344. 反转字符串}
\label{leetcode:344}

\subsection{题目}

编写一个函数,其作用是将输入的字符串反转过来。输入字符串以字符数组 char[] 的形式给出。

不要给另外的数组分配额外的空间,你必须原地修改输入数组、使用 O(1) 的额外空间解决这一问题。

你可以假设数组中的所有字符都是 ASCII 码表中的可打印字符。

\textbf{示例 1}:

\begin{verbatim}
  输入:["h","e","l","l","o"]
  输出:["o","l","l","e","h"]
\end{verbatim}

\textbf{示例 2}:

\begin{verbatim}
  输入:["H","a","n","n","a","h"]
  输出:["h","a","n","n","a","H"]
\end{verbatim}

\subsection{参考题解}

\subsubsection{Python 解法1}

\begin{verbatim}
class Solution:
  def reverseString(self, s: List[str]) -> None:
    """
    Do not return anything, modify s in-place instead.
    """
    i, j = 0, len(s) - 1
    while i < j:
      s[i], s[j] = s[j], s[i]
      i += 1
      j -= 1
\end{verbatim}

\subsubsection{Python 解法2}

\begin{verbatim}
class Solution:
  def reverseString(self, s: List[str]) -> None:
    """
    Do not return anything, modify s in-place instead.
    """
    s.reverse()
\end{verbatim}
