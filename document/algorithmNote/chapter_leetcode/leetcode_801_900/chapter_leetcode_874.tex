\newpage
\section{874. 模拟行走机器人}
\label{leetcode:874}

\subsection{题目}

机器人在一个无限大小的网格上行走,从点 (0, 0) 处开始出发,面向北方。
该机器人可以接收以下三种类型的命令:

\begin{enumerate}
  \item -2:向左转 90 度
  \item -1:向右转 90 度
  \item 1 <= x <= 9:向前移动 x 个单位长度
\end{enumerate}

在网格上有一些格子被视为障碍物。

第 i 个障碍物位于网格点  (obstacles[i][0], obstacles[i][1])

如果机器人试图走到障碍物上方,那么它将停留在障碍物的前一个网格方块上,
但仍然可以继续该路线的其余部分。

返回从原点到机器人的最大欧式距离的\textbf{平方}。

\textbf{示例 1}:

\begin{verbatim}
  输入: commands = [4,-1,3], obstacles = []
  输出: 25
  解释: 机器人将会到达 (3, 4)
\end{verbatim}

\textbf{示例 2}:

\begin{verbatim}
  输入: commands = [4,-1,4,-2,4], obstacles = [[2,4]]
  输出: 65
  解释: 机器人在左转走到 (1, 8) 之前将被困在 (1, 4) 处
\end{verbatim}

\textbf{提示}:

\begin{itemize}
  \item 0 <= commands.length <= 10000
  \item 0 <= obstacles.length <= 10000
  \item -30000 <= obstacle[i][0] <= 30000
  \item -30000 <= obstacle[i][1] <= 30000
  \item 答案保证小于 $2^{31}$
\end{itemize}

\subsection{参考题解}

\begin{verbatim}
/**
 * @param {number[]} commands
 * @param {number[][]} obstacles
 * @return {number}
 */
var robotSim = function(commands, obstacles) {
  let dirs = [ [0, 1], [1, 0], [0, -1], [-1, 0] ];
  let result = 0;
  let x = 0;
  let y = 0;
  let di = 0;
  let s = new Set();
  for (let i = 0; i < obstacles.length; i += 1) {
    let obstacle = obstacles[i];
    s.add('' + obstacle[0] + '_' + obstacle[1]);
  }
  for (let i = 0; i < commands.length; i += 1) {
    let command = commands[i];
    if (command === -1) {
      di = (di + 1) % 4;
    } else if (command === -2) {
      di = (di + 3) % 4;
    } else {
      for (let j = 0; j < command; j += 1) {
        const newX = x + dirs[di][0];
        const newY = y + dirs[di][1];
        if (!s.has('' + newX + '_' + newY)) {
          x = newX;
          y = newY;
          result = Math.max(result, x*x + y*y);
        }
      }
    }
  }
  return result;
};  
\end{verbatim}
