\newpage
\section{818. 赛车}
\label{leetcode:818}

\subsection{题目}

你的赛车起始停留在位置 0,速度为 +1,正行驶在一个无限长的数轴上。(车也可以
向负数方向行驶。)

你的车会根据一系列由 A(加速)和 R(倒车)组成的指令进行自动驾驶。

当车得到指令 \verb|"A"| 时, 将会做出以下操作: \\
position += speed, speed *= 2。

当车得到指令 \verb|"R"| 时, 将会做出以下操作: \\
如果当前速度是正数,则将车速调整为 speed = -1; \\
否则将车速调整为 speed = 1。 (当前所处位置不变。)

例如,当得到一系列指令 \verb|"AAR"| 后, \\
你的车将会走过位置 0->1->3->3, \\
并且速度变化为 1->2->4->-1。

现在给定一个目标位置,请给出能够到达目标位置的最短指令列表的长度。

\textbf{示例 1}:

\begin{verbatim}
  输入:
  target = 3
  输出: 2
  解释:
  最短指令列表为 "AA"
  位置变化为 0->1->3
\end{verbatim}

\textbf{示例 2}:

\begin{verbatim}
  输入:
  target = 6
  输出: 5
  解释:
  最短指令列表为 "AAARA"
  位置变化为 0->1->3->7->7->6
\end{verbatim}

\textbf{说明}:

\begin{verbatim}
  1 <= target(目标位置) <= 10000。
\end{verbatim}

\subsection{参考题解,BFS}

会超时

\begin{verbatim}
class Solution:
  def racecar(self, target: int) -> int:
    result = 0
    queue = collections.deque()
    queue.append((0, 1)) # position: 0, speed: 1
    while queue:
      curLevelSize = len(queue)
      for _ in range(curLevelSize):
        node = queue.popleft()
        curPosition, curSpeed = node[0], node[1]
        if curPosition == target:
          return result

        # process A
        newPosition = curPosition + curSpeed
        newSpeed = curSpeed * 2
        queue.append((newPosition, newSpeed))

        # process R
        newPosition = curPosition
        newSpeed = -1 if curSpeed > 0 else 1
        queue.append((newPosition, newSpeed))
      result += 1
    return result
\end{verbatim}

\subsection{参考题解}

从 source 朝着 target 一直执行 A 命令,一直走到
距离 target 最近的那个点和刚好超过 target 的那个点。
然后从这两个点递归下去。

\subsubsection{Python}

\begin{verbatim}
class Solution:
  def __init__(self):
    self.dp = [0] * 10001

  def racecar(self, target: int) -> int:
    if self.dp[target] > 0: return self.dp[target]
    n = math.floor(math.log2(target)) + 1
    if target + 1 == (1<<n):
      self.dp[target] = n
      return self.dp[target]
    # n 个 A 到达 2^n-1 位置,然后 R 反向,走完剩余
    # n + 1 就是 n 个 A 和一个 R
    # (1<<n) - 1 - target 就是 2^n-1 和 target 之间的距离
    self.dp[target] = n + 1 + self.racecar((1<<n) - 1 - target)
    # n-1 个 A 到达 2^(n-1)-1 位置,然后 R 反向走 m 个 A,再 R 反向,走完剩余
    # m 取值遍历 [0, n-1)
    for m in range(n - 1):
      cur = (n-1) + 1 + m + 1 + self.racecar(target - (1<<(n-1)) + (1<<m))
      self.dp[target] = min(self.dp[target], cur)
    return self.dp[target]
\end{verbatim}
