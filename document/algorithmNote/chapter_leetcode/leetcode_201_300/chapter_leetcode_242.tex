\newpage
\section{242. 有效的字母异位词}
\label{leetcode:242}

\subsection{题目}

给定两个字符串 s 和 t ,编写一个函数来判断 t 是否是 s 的字母异位词。

\textbf{示例 1}:

\begin{verbatim}
  输入: s = "anagram", t = "nagaram"
  输出: true
\end{verbatim}

\textbf{示例 2}:

\begin{verbatim}
  输入: s = "rat", t = "car"
  输出: false
\end{verbatim}

\textbf{说明}:\\
你可以假设字符串只包含小写字母。

\textbf{进阶}:\\
如果输入字符串包含 unicode 字符怎么办?你能否调整你的解法来应对这种情况?

\subsection{参考题解,JavaScript}

\begin{verbatim}
/**
 * @param {string} s
 * @param {string} t
 * @return {boolean}
 */
var isAnagram = function(s, t) {
  if (s.length !== t.length) { return false; }
  const m = new Array(26).fill(0);
  for (let i = 0; i < s.length; i += 1) {
    m[s.charCodeAt(i) - 97]++;
    m[t.charCodeAt(i) - 97]--;
  }
  return m.filter(e => e > 0).length === 0;
};
\end{verbatim}

\subsection{参考题解,Golang}

\begin{verbatim}
func isAnagram(s string, t string) bool {
  if len(s) != len(t) {
    return false
  }
  m := make([]int, 26)
  for i := range s {
    m[s[i]-97]++
    m[t[i]-97]--
  }
  for i := 0; i < len(m); i++ {
    if m[i] > 0 {
      return false
    }
  }
  return true
}
\end{verbatim}

\subsection{参考题解,Python}

\paragraph{Python 方法一}

\begin{verbatim}
class Solution:
  def isAnagram(self, s: str, t: str) -> bool:
    if len(s) != len(t): return False
    m = [0] * 26
    for i in range(len(s)):
      m[ord(s[i]) - 97] += 1
      m[ord(t[i]) - 97] -= 1
    return len(list(filter(lambda e: e > 0, m))) == 0
\end{verbatim}

\paragraph{Python 方法二}

\begin{verbatim}
class Solution:
  def isAnagram(self, s: str, t: str) -> bool:
    return collections.Counter(s) == collections.Counter(t)
\end{verbatim}
