\newpage
\section{297. 二叉树的序列化与反序列化}
\label{leetcode:297}

\subsection{题目}

序列化是将一个数据结构或者对象转换为连续的比特位的操作,
进而可以将转换后的数据存储在一个文件或者内存中,
同时也可以通过网络传输到另一个计算机环境,采取相反方式重构得到原数据。

请设计一个算法来实现二叉树的序列化与反序列化。
这里不限定你的序列 / 反序列化算法执行逻辑,
你只需要保证一个二叉树可以被序列化为一个字符串
并且将这个字符串反序列化为原始的树结构。

\textbf{示例}: 

\begin{verbatim}
你可以将以下二叉树:

    1
   / \
  2   3
     / \
    4   5

序列化为 "[1,2,3,null,null,4,5]"
\end{verbatim}

\textbf{提示}: 这与 LeetCode 目前使用的方式一致,
详情请参阅 LeetCode 序列化二叉树的格式。
你并非必须采取这种方式,你也可以采用其他的方法解决这个问题。

\textbf{说明}: 不要使用类的成员 / 全局 / 静态变量来存储状态,
你的序列化和反序列化算法应该是无状态的。

\subsection{参考题解}

\begin{verbatim}
/**
 * Definition for a binary tree node.
 * function TreeNode(val) {
 *     this.val = val;
 *     this.left = this.right = null;
 * }
 */

/**
 * Encodes a tree to a single string.
 *
 * @param {TreeNode} root
 * @return {string}
 */
var serialize = function(root) {
  let result = [];
  serializeDFS(root, result);
  return result.join(",");
};

function serializeDFS(root, result) {
  if (root === null) {
    result.push("null");
    return;
  }
  result.push(root.val);
  serializeDFS(root.left, result);
  serializeDFS(root.right, result);
}

/**
 * Decodes your encoded data to tree.
 *
 * @param {string} data
 * @return {TreeNode}
 */
var deserialize = function(data) {
  let array = data.split(",");
  return deserializeRecursion(array);
};

function deserializeRecursion(array) {
  if (array.length === 0) { return null; }

  let val = array[0];
  if (val === 'null') {
    array.shift();
    return null;
  }

  let node = new TreeNode(+val);
  array.shift();
  node.left = deserializeRecursion(array);
  node.right = deserializeRecursion(array);
  return node;
}

/**
 * Your functions will be called as such:
 * deserialize(serialize(root));
 */
\end{verbatim}
