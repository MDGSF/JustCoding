\newpage
\section{212. 单词搜索 II}
\label{leetcode:212}

\subsection{题目}

给定一个二维网格 \textbf{board} 和一个字典中的单词列表 \textbf{words},
找出所有同时在二维网格和字典中出现的单词。

单词必须按照字母顺序,通过相邻的单元格内的字母构成,
其中“相邻”单元格是那些水平相邻或垂直相邻的单元格。
同一个单元格内的字母在一个单词中不允许被重复使用。

\textbf{示例}:

\begin{verbatim}
  输入:
  words = ["oath","pea","eat","rain"] and board =
  [
    ['o','a','a','n'],
    ['e','t','a','e'],
    ['i','h','k','r'],
    ['i','f','l','v']
  ]

  输出: ["eat","oath"]
  说明:
  你可以假设所有输入都由小写字母 a-z 组成。
\end{verbatim}

\textbf{提示}:

\begin{itemize}
  \item 你需要优化回溯算法以通过更大数据量的测试。你能否早点停止回溯?
  \item 如果当前单词不存在于所有单词的前缀中,则可以立即停止回溯。
  什么样的数据结构可以有效地执行这样的操作?散列表是否可行?为什么?
  前缀树如何?如果你想学习如何实现一个基本的前缀树,请先查看这个问题: 实现Trie(前缀树)。
\end{itemize}

\subsection{参考题解}

\begin{verbatim}
class Solution:

  def findWords(self, board: List[List[str]], words: List[str]) -> List[str]:
    self.root = {}
    self.endOfWord = "#"
    for word in words:
      self.insert(word)

    rows = len(board)
    cols = len(board[0])
    visited = [[False for j in range(cols)] for i in range(rows)]
    result = set()
    for row in range(rows):
      for col in range(cols):
        self.dfs(self.root, board, row, col, "", result, visited)
    return list(result)

  def dfs(self, node, board, row, col, curStr, result, visited):
    if self.endOfWord in node:
      result.add(curStr)
    if not self.canWalk(board, row, col, visited):
      return
    char = board[row][col]
    if char not in node:
      return
    visited[row][col] = True
    for x, y in [(-1, 0), (1, 0), (0, -1), (0, 1)]:
      self.dfs(node[char], board, row+x, col+y, curStr+char, result, visited)
    visited[row][col] = False

  def canWalk(self, board, row, col, visited):
    if row < 0 or row >= len(board):
      return False
    if col < 0 or col >= len(board[0]):
      return False
    if visited[row][col]:
      return False
    return True

  def insert(self, word):
    node = self.root
    for char in word:
      node = node.setdefault(char, {})
    node[self.endOfWord] = self.endOfWord
\end{verbatim}


