\newpage
\section{213. 打家劫舍 II}
\label{leetcode:213}

\subsection{题目}

你是一个专业的小偷,计划偷窃沿街的房屋,每间房内都藏有一定的现金。
这个地方所有的房屋都\textbf{围成一圈},这意味着第一个房屋和最后一个房屋是紧挨着的。
同时,相邻的房屋装有相互连通的防盗系统,
\textbf{如果两间相邻的房屋在同一晚上被小偷闯入,系统会自动报警}。

给定一个代表每个房屋存放金额的非负整数数组,计算你
\textbf{在不触动警报装置的情况下},能够偷窃到的最高金额。

\textbf{示例 1}:

\begin{verbatim}
  输入: [2,3,2]
  输出: 3
  解释: 你不能先偷窃 1 号房屋(金额 = 2),然后偷窃 3 号房屋(金额 = 2),
    因为他们是相邻的。
\end{verbatim}

\textbf{示例 2}:

\begin{verbatim}
  输入: [1,2,3,1]
  输出: 4
  解释: 你可以先偷窃 1 号房屋(金额 = 1),然后偷窃 3 号房屋(金额 = 3)。
       偷窃到的最高金额 = 1 + 3 = 4 。
\end{verbatim}

\subsection{参考题解}

环状排列房间可以化为两个单排排列房间的子问题:

\begin{itemize}
  \item 不偷第一个房间,最大金额是 p1
  \item 不偷最后一个房间,最大金额是 p2
\end{itemize}

那么偷窃环状排列的房间最大金额为 max(p1, p2)

单排排列房间问题详细见 \hyperref[leetcode:198]{198. 打家劫舍}

\begin{verbatim}
/**
* @param {number[]} nums
* @return {number}
*/
var rob = function(nums) {
  if (nums.length === 0) { return 0; }
  if (nums.length === 1) { return nums[0]; }
  let sub1 = rob1(nums, 1, nums.length-1);
  let sub2 = rob1(nums, 0, nums.length-2);
  return Math.max(sub1, sub2);
};

var rob1 = function(nums, start, end) {
  if (start > end) { return 0; }
  if (start === end) { return nums[start]; }
  f0 = nums[start];
  f1 = Math.max(nums[start], nums[start+1]);
  for (let i = start+2; i <= end; i += 1) {
    [f0, f1] = [f1, Math.max(nums[i]+f0, f1)];
  }
  return f1;
};
\end{verbatim}
