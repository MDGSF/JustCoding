\newpage
\section{771. 宝石与石头}
\label{leetcode:771}

\subsection{题目}

给定字符串J 代表石头中宝石的类型,和字符串 S代表你拥有的石头。 
S 中每个字符代表了一种你拥有的石头的类型,你想知道你拥有的石头中有多少是宝石。

J 中的字母不重复,J 和 S中的所有字符都是字母。字母区分大小写,因此``a''和``A''是不同类型的石头。

\textbf{示例 1}:

\begin{verbatim}
  输入: J = "aA", S = "aAAbbbb"
  输出: 3
\end{verbatim}

\textbf{示例 2}:

\begin{verbatim}
  输入: J = "z", S = "ZZ"
  输出: 0
\end{verbatim}

\textbf{注意}:

S 和 J 最多含有50个字母。J 中的字符不重复。

\subsection{参考题解}

\subsubsection{Python}

\begin{verbatim}
class Solution:
  def numJewelsInStones(self, J: str, S: str) -> int:
    Jset = set(J)
    return sum(c in Jset for c in S)
\end{verbatim}
