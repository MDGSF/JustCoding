\newpage
\section{1. 两数之和}
\label{leetcode:1}

\subsection{题目}

给定一个整数数组 nums 和一个目标值 target,
请你在该数组中找出和为目标值的那\textbf{两个}整数,并返回他们的数组下标。

你可以假设每种输入只会对应一个答案。但是,你不能重复利用这个数组中同样的元素。

\textbf{示例}:

\begin{verbatim}
  给定 nums = [2, 7, 11, 15], target = 9

  因为 nums[0] + nums[1] = 2 + 7 = 9
  所以返回 [0, 1]
\end{verbatim}

\subsection{参考题解,暴力法}

直接两重循环,遍历两个数的所有组合情况。
因为两重循环,所以时间复杂度为 O(n$^{2}$)。
因为没有而外开辟空间,所以空间复杂度为 O(1)。

\subsubsection{JavaScript}

\begin{verbatim}
/**
 * @param {number[]} nums
 * @param {number} target
 * @return {number[]}
 */
var twoSum = function(nums, target) {
  for (let i = 0; i < nums.length; i += 1) {
    for (let j = i + 1; j < nums.length; j += 1) {
      if (nums[i] + nums[j] === target) {
        return [i, j];
      }
    }
  }
};
\end{verbatim}

\subsubsection{Golang}

\begin{verbatim}
func twoSum(nums []int, target int) []int {
  for i := 0; i < len(nums); i++ {
    for j := i + 1; j < len(nums); j++ {
      if nums[i]+nums[j] == target {
        return []int{i, j}
      }
    }
  }
  return nil
}
\end{verbatim}

\subsection{参考题解,哈希表,两重循环}

因为 numA + numB = target,所以 numA = target - numB。
那我们可以考虑把 numA 的所有可能保存到哈希表,
然后第二次循环 numB 的时候,直接在哈希表中查找。

时间复杂度: O(n)。\\
空间复杂度: O(n)。

\subsubsection{JavaScript}

\begin{verbatim}
/**
 * @param {number[]} nums
 * @param {number} target
 * @return {number[]}
 */
var twoSum = function(nums, target) {
  const m = new Map();
  for (let i = 0; i < nums.length; i += 1) {
    m.set(nums[i], i);
  }
  for (let i = 0; i < nums.length; i += 1) {
    const num = nums[i];
    const peer = target - num;
    if (m.has(peer) && m.get(peer) !== i) {
      return [i, m.get(peer)];
    }
  }
};
\end{verbatim}

\subsubsection{Golang}

\begin{verbatim}
func twoSum(nums []int, target int) []int {
  m := make(map[int]int)
  for i := 0; i < len(nums); i++ {
    m[nums[i]] = i
  }
  for i := 0; i < len(nums); i++ {
    num := nums[i]
    peer := target - num
    peerIdx, ok := m[peer]
    if ok && peerIdx != i {
      return []int{i, peerIdx}
    }
  }
  return nil
}
\end{verbatim}

\subsection{参考题解,哈希表,一重循环}

这种方法是在上一种的基础上再次优化,减少了一重循环,
不过时间复杂度和空间复杂度还是一样的。

时间复杂度: O(n)。\\
空间复杂度: O(n)。

\subsubsection{JavaScript}

\begin{verbatim}
/**
 * @param {number[]} nums
 * @param {number} target
 * @return {number[]}
 */
var twoSum = function(nums, target) {
  const m = new Map();
  for (let i = 0; i < nums.length; i += 1) {
    const num = nums[i];
    const peer = target - num;
    if (m.has(peer)) {
      return [m.get(peer), i];
    }
    m.set(num, i);
  }
};
\end{verbatim}

\subsubsection{Golang}

\begin{verbatim}
func twoSum(nums []int, target int) []int {
  m := make(map[int]int)
  for i := 0; i < len(nums); i++ {
    num := nums[i]
    peer := target - num
    if peerIdx, ok := m[peer]; ok {
      return []int{peerIdx, i}
    }
    m[num] = i
  }
  return nil
}
\end{verbatim}

\subsection{题目小结}

通过上面 3 种题解,你会发现这就是非常明显的空间换时间。
在工程项目中也是非常常用的方法,所以要熟练掌握哈希表。

\subsection{题目拓展}

如果题目要求返回的不是下标,而是两个整数的值,
那你还有其他方法吗?

答案是\textbf{双指针两头夹逼}的方法。该方法要求数组是有序的,
所以需要先对数组进行排序。如果是要返回下标的话,排序了
之后,下标就改变了,就不适合用\textbf{双指针两头夹逼}的方法。

下面给出如果是返回整数的值的参考代码:

\begin{verbatim}
var twoSum = function(nums, target) {
  nums.sort((a, b) => a - b);
  let [start, end] = [0, nums.length - 1];
  while (start < end) {
    const sum = nums[start] + nums[end];
    if (sum < target) {
      start += 1;
    } else if (sum > target) {
      end -= 1;
    } else {
      return [nums[start], nums[end]];
    }
  }
};
\end{verbatim}
