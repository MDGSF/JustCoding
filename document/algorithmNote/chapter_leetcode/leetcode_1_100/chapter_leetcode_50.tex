\newpage
\section{50. Pow(x, n)}
\label{leetcode:50}

\subsection{题目}

实现 pow(x, n) ,即计算 x 的 n 次幂函数。

\textbf{示例 1}:

\begin{verbatim}
  输入: 2.00000, 10
  输出: 1024.00000
\end{verbatim}

\textbf{示例 2}:

\begin{verbatim}
  输入: 2.10000, 3
  输出: 9.26100
\end{verbatim}

\textbf{示例 3}:

\begin{verbatim}
  输入: 2.00000, -2
  输出: 0.25000
  解释: 2^-2 = 1/2^2 = 1/4 = 0.25
\end{verbatim}

\textbf{说明}:

\begin{verbatim}
  -100.0 < x < 100.0
  n 是 32 位有符号整数,其数值范围是 [−2^31, 2^31 − 1] 。
\end{verbatim}

\subsection{参考题解,暴力法}

直接循环 n 次,即可得到答案。会超出时间限制。\\
时间复杂度为 O(n)。

\begin{verbatim}
/**
 * @param {number} x
 * @param {number} n
 * @return {number}
 */
var myPow = function(x, n) {
  if (n < 0) {
    x = 1 / x;
    n = -n;
  }

  let result = 1;
  for (let i = 0; i < n; i += 1) {
    result *= x;
  }
  return result;
};
\end{verbatim}

\subsection{参考题解,分治,递归1}

当 n 为偶数时, $x^{n} = x^{n/2} * x^{n/2}$ \\
当 n 为奇数时, $x^{n} = x^{(n-1)/2} * x^{(n-1)/2} * x$ 

\begin{verbatim}
/**
 * @param {number} x
 * @param {number} n
 * @return {number}
 */
var myPow = function(x, n) {
  if (n < 0) {
    x = 1 / x;
    n = -n;
  }
  if (n === 0) {
    return 1;
  }

  if (n % 2 === 0) {
    let sub = myPow(x, n / 2);
    return sub * sub;
  }
  let sub = myPow(x, Math.floor(n / 2));
  return sub * sub * x;
};
\end{verbatim}

\subsection{参考题解,分治,递归2}

这个方法和上面一种是一样的思路,只是代码实现有些微区别。

\begin{verbatim}
/**
 * @param {number} x
 * @param {number} n
 * @return {number}
 */
var myPow = function(x, n) {
  if (n < 0) { return 1 / myPow(x, -n); }
  if (n === 0) { return 1; }
  if (n % 2 === 0) { return myPow(x * x, n / 2); }
  return x * myPow(x, n - 1);
};
\end{verbatim}

\subsection{参考题解,位运算}

\begin{verbatim}
/**
 * @param {number} x
 * @param {number} n
 * @return {number}
 */
var myPow = function(x, n) {
  if (n < 0) {
    x = 1 / x;
    n = -n;
  }
  let pow = 1;
  while (n > 0) {
    if ((n & 1) === 1) {
      pow *= x;
    }
    x *= x;
    n >>>= 1;
  }
  return pow;
};
\end{verbatim}
