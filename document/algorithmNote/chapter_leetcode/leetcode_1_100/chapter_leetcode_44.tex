\newpage
\section{44. 通配符匹配}
\label{leetcode:44}

\subsection{题目}

给定一个字符串 (s) 和一个字符模式 (p) ,实现一个支持 \verb|'?'| 和 \verb|'*'| 的通配符匹配。

\begin{itemize}
  \item \verb|'?'| 可以匹配任何单个字符。
  \item \verb|'*'| 可以匹配任意字符串(包括空字符串)。
\end{itemize}

两个字符串\textbf{完全匹配}才算匹配成功。

\textbf{说明}:

\begin{itemize}
  \item s 可能为空,且只包含从 a-z 的小写字母。
  \item p 可能为空,且只包含从 a-z 的小写字母,以及字符 ? 和 *。
\end{itemize}

\textbf{示例 1}:

\begin{verbatim}
  输入:
  s = "aa"
  p = "a"
  输出: false
  解释: "a" 无法匹配 "aa" 整个字符串。
\end{verbatim}

\textbf{示例 2}:

\begin{verbatim}
  输入:
  s = "aa"
  p = "*"
  输出: true
  解释: '*' 可以匹配任意字符串。
\end{verbatim}

\textbf{示例 3}:

\begin{verbatim}
  输入:
  s = "cb"
  p = "?a"
  输出: false
  解释: '?' 可以匹配 'c', 但第二个 'a' 无法匹配 'b'。
\end{verbatim}

\textbf{示例 4}:

\begin{verbatim}
  输入:
  s = "adceb"
  p = "*a*b"
  输出: true
  解释: 第一个 '*' 可以匹配空字符串, 第二个 '*' 可以匹配字符串 "dce".
\end{verbatim}

\textbf{示例 5}:

\begin{verbatim}
  输入:
  s = "acdcb"
  p = "a*c?b"
  输入: false
\end{verbatim}

\subsection{参考题解,傻递归}

\subsubsection{Python}

\begin{verbatim}
class Solution:
  def isMatch(self, s: str, p: str) -> bool:
    if len(p) == 0:
      return len(s) == 0
    if p[0] == '*':
      return self.isMatch(s, p[1:]) or (len(s) != 0) and self.isMatch(s[1:], p)
    else:
      firstMatch = (len(s) != 0) and (p[0] == s[0] or p[0] == '?')
      return firstMatch and self.isMatch(s[1:], p[1:])
\end{verbatim}

\subsection{参考题解,递归+缓存}

\subsubsection{Python}

\begin{verbatim}
class Solution:
  def isMatch(self, s: str, p: str) -> bool:
    m = {}
    def dfs(i, j):
      if (i, j) in m:
        return m[(i, j)]
      if len(p) == j:
        return len(s) == i
      if p[j] == '*':
        result = dfs(i, j + 1) or i < len(s) and dfs(i + 1, j)
      else:
        firstMatch = (i < len(s)) and (p[j] == s[i] or p[j] == '?')
        result = firstMatch and dfs(i + 1, j + 1)
      m[(i, j)] = result
      return result
    return dfs(0, 0)
\end{verbatim}
