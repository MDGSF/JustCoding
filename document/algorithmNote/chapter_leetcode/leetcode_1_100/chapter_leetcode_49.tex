\newpage
\section{49. 字母异位词分组}
\label{leetcode:49}

\subsection{题目}

给定一个字符串数组,将字母异位词组合在一起。字母异位词指字母相同,但排列不同的字符串。

\textbf{示例}:

\begin{verbatim}
  输入: ["eat", "tea", "tan", "ate", "nat", "bat"],
  输出:
  [
    ["ate","eat","tea"],
    ["nat","tan"],
    ["bat"]
  ]
\end{verbatim}

\textbf{说明}:

\begin{verbatim}
  所有输入均为小写字母。
  不考虑答案输出的顺序。
\end{verbatim}

\subsection{参考题解,JavaScript}

把字符串的每个字符排序之后,那字母异位词排序之后就会相同。

比如:

\begin{enumerate}
  \item ``ate'' 排序之后得到 ``aet''
  \item ``eat'' 排序之后得到 ``aet''
  \item ``tea'' 排序之后得到 ``aet''
\end{enumerate}

那么 ``ate'',``eat'' 和 ``tea'' 就是字母异位词。

\begin{verbatim}
/**
 * @param {string[]} strs
 * @return {string[][]}
 */
var groupAnagrams = function(strs) {
  const m = {};
  for (let i = 0; i < strs.length; i += 1) {
    const str = strs[i];
    const key = str.split("").sort().join("");
    if (key in m) {
      m[key].push(str);
    } else {
      m[key] = [str];
    }
  }

  const result = [];
  for (let key in m) {
    result.push(m[key]);
  }
  return result;
};
\end{verbatim}

\subsection{参考题解,Golang}

\begin{verbatim}
func groupAnagrams(strs []string) [][]string {
  m := make(map[string][]string)
  for _, str := range strs {
    letters := strings.Split(str, "")
    sort.Strings(letters)
    key := strings.Join(letters, "")
    if _, ok := m[key]; ok {
      m[key] = append(m[key], str)
    } else {
      m[key] = []string{str}
    }
  }

  result := make([][]string, 0)
  for key := range m {
    result = append(result, m[key])
  }
  return result
}
\end{verbatim}
