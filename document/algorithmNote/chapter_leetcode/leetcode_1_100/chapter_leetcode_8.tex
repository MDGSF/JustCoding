\newpage
\section{8. 字符串转换整数 (atoi)}
\label{leetcode:8}

\subsection{题目}

请你来实现一个 atoi 函数,使其能将字符串转换成整数。

首先,该函数会根据需要丢弃无用的开头空格字符,直到寻找到第一个非空格的字符为止。

当我们寻找到的第一个非空字符为正或者负号时,则将该符号与之后面尽可能多的连续数字组合起来,
作为该整数的正负号;假如第一个非空字符是数字,则直接将其与之后连续的数字字符组合起来,形成整数。

该字符串除了有效的整数部分之后也可能会存在多余的字符,这些字符可以被忽略,它们对于函数不应该造成影响。

注意:假如该字符串中的第一个非空格字符不是一个有效整数字符、字符串为空或字符串仅包含空白字符时,
则你的函数不需要进行转换。

在任何情况下,若函数不能进行有效的转换时,请返回 0。

\textbf{说明}:

假设我们的环境只能存储 32 位大小的有符号整数,那么其数值范围为 [−$2^{31}$,  $2^{31}$ − 1]。
如果数值超过这个范围,请返回  INT\_MAX ($2^{31}$ − 1) 或 INT\_MIN (−$2^{31}$) 。

\textbf{示例 1}:

\begin{verbatim}
  输入: "42"
  输出: 42
\end{verbatim}

\textbf{示例 2}:

\begin{verbatim}
  输入: "   -42"
  输出: -42
  解释: 第一个非空白字符为 '-', 它是一个负号。
       我们尽可能将负号与后面所有连续出现的数字组合起来,最后得到 -42 。
\end{verbatim}

\textbf{示例 3}:

\begin{verbatim}
  输入: "4193 with words"
  输出: 4193
  解释: 转换截止于数字 '3' ,因为它的下一个字符不为数字。
\end{verbatim}

\textbf{示例 4}:

\begin{verbatim}
  输入: "words and 987"
  输出: 0
  解释: 第一个非空字符是 'w', 但它不是数字或正、负号。
      因此无法执行有效的转换。
\end{verbatim}

\textbf{示例 5}:

\begin{verbatim}
  输入: "-91283472332"
  输出: -2147483648
  解释: 数字 "-91283472332" 超过 32 位有符号整数范围。
       因此返回 INT_MIN (−2^31) 。
\end{verbatim}

\subsection{参考题解}

\subsubsection{Python}

\begin{verbatim}
class Solution:
  def myAtoi(self, s: str) -> int:
    if len(s) == 0: return 0
    ls = list(s.strip())
    if len(ls) == 0: return 0
    sign = -1 if ls[0] == '-' else 1
    if ls[0] in ['-', '+']: del ls[0]
    ret, i = 0, 0
    while i < len(ls) and ls[i].isdigit():
      ret = ret * 10 + ord(ls[i]) - ord('0')
      i += 1
    return max(-2**31, min(sign * ret, 2**31 - 1))
\end{verbatim}
