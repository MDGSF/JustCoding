\newpage
\section{69. x 的平方根}
\label{leetcode:69}

\subsection{题目}

实现 int sqrt(int x) 函数。

计算并返回 x 的平方根,其中 x 是非负整数。

由于返回类型是整数,结果只保留整数的部分,小数部分将被舍去。

\textbf{示例 1}:

\begin{verbatim}
  输入: 4
  输出: 2
\end{verbatim}

\textbf{示例 2}:

\begin{verbatim}
  输入: 8
  输出: 2
  说明: 8 的平方根是 2.82842..., 
       由于返回类型是整数,小数部分将被舍去。
\end{verbatim}

\subsection{参考题解,二分法}

\begin{verbatim}
/**
 * @param {number} x
 * @return {number}
 */
var mySqrt = function(x) {
  return Math.floor(MySqrt(x, 0.01));
};

function MySqrt(x, precision) {
  if (x === 0 || x === 1) {
    return x;
  }

  let left = 0;
  let right = x;
  while (left <= right) {
    let mid = (left + right) / 2;
    let cur = mid * mid;
    if (cur - x >= -precision && cur - x <= precision) {
      return mid;
    } else if (cur > x) {
      right = mid;
    } else {
      left = mid;
    }
  }
  return null;
}
\end{verbatim}

\subsection{参考题解,牛顿迭代法}

\begin{verbatim}
/**
* @param {number} x
* @return {number}
*/
var mySqrt = function(x) {
  let r = x;
  while (r * r > x) {
    r = Math.floor((r + Math.floor(x / r)) / 2);
  }
  return r;
};
\end{verbatim}

\subsection{扩展,雷神之锤}

\begin{verbatim}
float InvSqrt (float x){
  float xhalf = 0.5f*x;
  int i = *(int*)&x;
  i = 0x5f3759df - (i>>1);
  x = *(float*)&i;
  x = x*(1.5f - xhalf*x*x);
  return x;
}
\end{verbatim}
