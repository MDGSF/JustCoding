\newpage
\section{1143. 最长公共子序列}
\label{leetcode:11431}

\subsection{题目}

给定两个字符串 text1 和 text2,返回这两个字符串的最长公共子序列。

一个字符串的 子序列 是指这样一个新的字符串:它是由原字符串在不改变字符的
相对顺序的情况下删除某些字符(也可以不删除任何字符)后组成的新字符串。
例如,"ace" 是 "abcde" 的子序列,但 "aec" 不是 "abcde" 的子序列。
两个字符串的「公共子序列」是这两个字符串所共同拥有的子序列。

若这两个字符串没有公共子序列,则返回 0。

\textbf{示例 1}:

\begin{verbatim}
  输入:text1 = "abcde", text2 = "ace"
  输出:3
  解释:最长公共子序列是 "ace",它的长度为 3。
\end{verbatim}

\textbf{示例 2}:

\begin{verbatim}
  输入:text1 = "abc", text2 = "abc"
  输出:3
  解释:最长公共子序列是 "abc",它的长度为 3。
\end{verbatim}

\textbf{示例 3}:

\begin{verbatim}
  输入:text1 = "abc", text2 = "def"
  输出:0
  解释:两个字符串没有公共子序列,返回 0。
\end{verbatim}

\textbf{提示}:

\begin{verbatim}
  1 <= text1.length <= 1000
  1 <= text2.length <= 1000
  输入的字符串只含有小写英文字符。
\end{verbatim}

\subsection{参考题解}

\subsubsection{Python}

\begin{verbatim}
class Solution:
  def longestCommonSubsequence(self, text1: str, text2: str) -> int:
    if not text1 or not text2: return 0
    rows, cols = len(text1) + 1, len(text2) + 1
    dp = [[0] * cols for _ in range(rows)]
    for row in range(1, rows):
      for col in range(1, cols):
        if text1[row-1] == text2[col-1]:
          dp[row][col] = dp[row-1][col-1] + 1
        else:
          dp[row][col] = max(dp[row-1][col], dp[row][col-1])
    return dp[-1][-1]
\end{verbatim}

