\newpage
\section{547. 朋友圈}
\label{leetcode:547}

\subsection{题目}

班上有 N 名学生。其中有些人是朋友,有些则不是。他们的友谊具有是传递性。
如果已知 A 是 B 的朋友,B 是 C 的朋友,那么我们可以认为 A 也是 C 的朋友。
所谓的朋友圈,是指所有朋友的集合。

给定一个 N * N 的矩阵 M,表示班级中学生之间的朋友关系。
如果M[i][j] = 1,表示已知第 i 个和 j 个学生互为朋友关系,否则为不知道。
你必须输出所有学生中的已知的朋友圈总数。

\textbf{示例 1}:

\begin{verbatim}
  输入:
  [[1,1,0],
   [1,1,0],
   [0,0,1]]
  输出: 2
  说明:已知学生0和学生1互为朋友,他们在一个朋友圈。
  第2个学生自己在一个朋友圈。所以返回2。
\end{verbatim}

\textbf{示例 2}:

\begin{verbatim}
  输入:
  [[1,1,0],
   [1,1,1],
   [0,1,1]]
  输出: 1
  说明:已知学生0和学生1互为朋友,学生1和学生2互为朋友,所以学
  生0和学生2也是朋友,所以他们三个在一个朋友圈,返回1。
\end{verbatim}

\textbf{注意}:

\begin{verbatim}
  N 在[1,200]的范围内。
  对于所有学生,有M[i][i] = 1。
  如果有M[i][j] = 1,则有M[j][i] = 1。
\end{verbatim}

\subsection{参考题解}

\begin{verbatim}
class Solution:
  def findCircleNum(self, M: List[List[int]]) -> int:
    if not M: return 0
    n = len(M)
    p = [i for i in range(n)]

    for i in range(n):
      for j in range(n):
        if M[i][j] == 1:
          self.union(p, i, j)

    return len(set([self.parent(p, i) for i in range(n)]))

  def union(self, p, i, j):
    p1 = self.parent(p, i)
    p2 = self.parent(p, j)
    p[p1] = p2

  def parent(self, p, i):
    root = i
    while p[root] != root:
      root = p[root]
    while p[i] != i:
      x = i; i = p[i]; p[x] = root
    return root
\end{verbatim}
