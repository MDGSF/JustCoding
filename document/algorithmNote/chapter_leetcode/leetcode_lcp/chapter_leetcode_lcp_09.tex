\newpage
\section{LCP 09. 最小跳跃次数}
\label{leetcode:lcp_09}

\subsection{题目}

为了给刷题的同学一些奖励,力扣团队引入了一个弹簧游戏机。
游戏机由 N 个特殊弹簧排成一排,编号为 0 到 N-1。
初始有一个小球在编号 0 的弹簧处。若小球在编号为 i 的弹簧处,
通过按动弹簧,可以选择把小球向右弹射 jump[i] 的距离,
或者向左弹射到任意左侧弹簧的位置。也就是说,在编号为 i 弹簧处按动弹簧,
小球可以弹向 0 到 i-1 中任意弹簧或者 i+jump[i] 的弹簧(若 i+jump[i]>=N ,则表示小球弹出了机器)。
小球位于编号 0 处的弹簧时不能再向左弹。

为了获得奖励,你需要将小球弹出机器。请求出最少需要按动多少次弹簧,
可以将小球从编号 0 弹簧弹出整个机器,即向右越过编号 N-1 的弹簧。

\textbf{示例 1}:

\begin{verbatim}
  输入:jump = [2, 5, 1, 1, 1, 1]
  输出:3
  解释:小 Z 最少需要按动 3 次弹簧,小球依次到达的顺序为
      0 -> 2 -> 1 -> 6,最终小球弹出了机器。
\end{verbatim}

\textbf{限制}:

\begin{verbatim}
  1 <= jump.length <= 10^6
  1 <= jump[i] <= 10000
\end{verbatim}

\subsection{参考题解}

\subsubsection{Python}

\begin{verbatim}
class Solution:
  def minJump(self, jump: List[int]) -> int:
    dp = [0] * len(jump)
    for i in range(len(jump)-1, -1, -1):
      if i  + jump[i] >= len(jump):
        dp[i] = 1
      else:
        dp[i] = dp[i + jump[i]] + 1
      j = i + 1
      while j < len(jump) and j < i + jump[i] and dp[j] > dp[i]:
        dp[j] = min(dp[j], dp[i] + 1)
        j += 1
    return dp[0]
\end{verbatim}
