\newpage
\section{122. 买卖股票的最佳时机 II}
\label{leetcode:122}

\subsection{题目}

给定一个数组,它的第 i 个元素是一支给定股票第 i 天的价格。

设计一个算法来计算你所能获取的最大利润。你可以尽可能地完成更多的交易(多次买卖一支股票)。

注意:你不能同时参与多笔交易(你必须在再次购买前出售掉之前的股票)。

\textbf{示例 1}:

\begin{verbatim}
  输入: [7,1,5,3,6,4]
  输出: 7
  解释: 在第 2 天(股票价格 = 1)的时候买入,在第 3 天(股票价格 = 5)的时候
  卖出, 这笔交易所能获得利润 = 5-1 = 4 。
    随后,在第 4 天(股票价格 = 3)的时候买入,在第 5 天(股票价格 = 6)的
  时候卖出, 这笔交易所能获得利润 = 6-3 = 3 。
\end{verbatim}

\textbf{示例 2}:

\begin{verbatim}
  输入: [1,2,3,4,5]
  输出: 4
  解释: 在第 1 天(股票价格 = 1)的时候买入,在第 5 天 (股票价格 = 5)的时
  候卖出, 这笔交易所能获得利润 = 5-1 = 4 。
    注意你不能在第 1 天和第 2 天接连购买股票,之后再将它们卖出。
    因为这样属于同时参与了多笔交易,你必须在再次购买前出售掉之前的股票。
\end{verbatim}

\textbf{示例 3}:

\begin{verbatim}
  输入: [7,6,4,3,1]
  输出: 0
  解释: 在这种情况下, 没有交易完成, 所以最大利润为 0。
\end{verbatim}

\subsection{参考题解,动态规划}

\begin{verbatim}
  dp[i][0] = max(dp[i-1][0], dp[i-1][1] + prices[i])
  dp[i][1] = max(dp[i-1][1], dp[i-1][0] - prices[i])
\end{verbatim}

\subsubsection{Python}

\begin{verbatim}
class Solution:
  def maxProfit(self, prices: List[int]) -> int:
    dp_i_0, dp_i_1 = 0, float('-inf')
    for i in range(len(prices)):
      pre_i_0, pre_i_1 = dp_i_0, dp_i_1
      dp_i_0 = max(pre_i_0, pre_i_1 + prices[i])
      dp_i_1 = max(pre_i_1, pre_i_0 - prices[i])
    return dp_i_0
\end{verbatim}

\subsection{参考题解,DFS}

每天都可以买一股,或者是卖一股,所以时间复杂度是 O(2$^{n}$)。

可以遍历到所有的情况。

\subsection{参考题解,贪心法}

只要后一天的价格比前一天高,我们就在前一天买入,后一天卖出。

\begin{verbatim}
/**
 * @param {number[]} prices
 * @return {number}
 */
var maxProfit = function(prices) {
  let result = 0;
  for (let i = 1; i < prices.length; i += 1) {
    const diff = prices[i] - prices[i - 1];
    if (diff > 0) {
      result += diff;
    }
  }
  return result;
};
\end{verbatim}
