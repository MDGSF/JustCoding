\newpage
\section{127. 单词接龙}
\label{leetcode:127}

\subsection{题目}

给定两个单词(beginWord 和 endWord)和一个字典,
找到从 beginWord 到 endWord 的最短转换序列的长度。
转换需遵循如下规则:

\begin{itemize}
  \item 每次转换只能改变一个字母。
  \item 转换过程中的中间单词必须是字典中的单词。
\end{itemize}

\textbf{说明}:

\begin{enumerate}
  \item 如果不存在这样的转换序列,返回 0。
  \item 所有单词具有相同的长度。
  \item 所有单词只由小写字母组成。
  \item 字典中不存在重复的单词。
  \item 你可以假设 beginWord 和 endWord 是非空的,且二者不相同。
\end{enumerate}

\textbf{示例 1}:

\begin{verbatim}
  输入:
  beginWord = "hit",
  endWord = "cog",
  wordList = ["hot","dot","dog","lot","log","cog"]

  输出: 5

  解释: 一个最短转换序列是 "hit" -> "hot" -> "dot" -> "dog" -> "cog",
      返回它的长度 5。
\end{verbatim}

\textbf{示例 2}:

\begin{verbatim}
  输入:
  beginWord = "hit"
  endWord = "cog"
  wordList = ["hot","dot","dog","lot","log"]

  输出: 0

  解释: endWord "cog" 不在字典中,所以无法进行转换。
\end{verbatim}

\subsection{参考题解,BFS}

\begin{verbatim}
/**
 * @param {string} beginWord
 * @param {string} endWord
 * @param {string[]} wordList
 * @return {number}
 */
var ladderLength = function(beginWord, endWord, wordList) {
  const charSet = generateCharSet();
  let wordSet = new Set(wordList);
  let visited = new Set();
  let queue = [];
  visited.add(beginWord);
  queue.push(beginWord);
  let level = 0;
  while (queue.length > 0) {
    let levelSize = queue.length;
    while (levelSize-- > 0) {
      let cur = queue.shift();
      if (cur === endWord) {
        return level + 1;
      }
      let curCharArray = cur.split("");
      for (let i = 0; i < curCharArray.length; i += 1) {
        let old = curCharArray[i];
        for (let j = 0; j < charSet.length; j += 1) {
          curCharArray[i] = charSet[j];
          let next = curCharArray.join("");
          if (!visited.has(next) && wordSet.has(next)) {
            visited.add(next);
            queue.push(next);
          }
        }
        curCharArray[i] = old;
      }
    }
    level += 1;
  }
  return 0;
};

function generateCharSet() {
  let charSet = [];
  let start = 'a'.charCodeAt(0);
  let end = start + 26;
  for (let i = start; i < end; i += 1) {
    charSet.push(String.fromCharCode(i));
  }
  return charSet;
}
\end{verbatim}

\subsection{参考题解,双向BFS}

中间的 3 重 for 循环,这里解释下,beginSet 和 endSet 就是
保存当前层的节点。然后需要遍历当前层的所有节点,每个节点就是一个
单词,然后遍历每个单词的每个字符,把每个字符分别挨个替换为 a - z 
的每一个。

\begin{verbatim}
/**
 * @param {string} beginWord
 * @param {string} endWord
 * @param {string[]} wordList
 * @return {number}
 */
var ladderLength = function(beginWord, endWord, wordList) {
  let wordSet = new Set(wordList);
  if (!wordSet.has(endWord)) { return false; }
  let beginSet = new Set();
  let endSet = new Set();
  let visited = new Set();
  let level = 1;
  beginSet.add(beginWord);
  endSet.add(endWord);
  while (beginSet.size > 0 && endSet.size > 0) {
    if (beginSet.size > endSet.size) {
      [beginSet, endSet] = [endSet, beginSet];
    }
    let temp = new Set();
    for (let word of beginSet) {
      let wordCharArray = word.split("");
      for (let i = 0; i < wordCharArray.length; i += 1) {
        let old = wordCharArray[i];
        for (let j = 0; j < 26; j += 1) {
          wordCharArray[i] = String.fromCharCode(97+j);
          let next = wordCharArray.join("");
          if (endSet.has(next)) {
            return level + 1;
          }
          if (!visited.has(next) && wordSet.has(next)) {
            visited.add(next);
            temp.add(next);
          }
        }
        wordCharArray[i] = old;
      }
    }
    beginSet = temp;
    level += 1;
  }
  return 0;
};
\end{verbatim}
