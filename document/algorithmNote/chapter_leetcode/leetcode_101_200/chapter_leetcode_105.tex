\newpage
\section{105. 从前序与中序遍历序列构造二叉树}
\label{leetcode:105}

\subsection{题目}

根据一棵树的前序遍历与中序遍历构造二叉树。

\textbf{注意}: \\
你可以假设树中没有重复的元素。

例如,给出

\begin{verbatim}
  前序遍历 preorder = [3,9,20,15,7]
  中序遍历 inorder = [9,3,15,20,7]
\end{verbatim}

返回如下的二叉树:

\begin{verbatim}
   3
  / \
 9  20
   /  \
  15   7
\end{verbatim}

\subsection{参考题解,JavaScript}

JavaScript 的 slice 会拷贝数组。

\begin{verbatim}
/**
 * Definition for a binary tree node.
 * function TreeNode(val) {
 *     this.val = val;
 *     this.left = this.right = null;
 * }
 */
/**
 * @param {number[]} preorder
 * @param {number[]} inorder
 * @return {TreeNode}
 */
var buildTree = function(preorder, inorder) {
  if (preorder.length === 0 || inorder.length === 0) {
    return null;
  }
  let val = preorder[0];
  let idx = inorder.indexOf(val);
  let node = new TreeNode(val);
  node.left = buildTree(preorder.slice(1, 1+idx), inorder.slice(0, idx));
  node.right = buildTree(preorder.slice(1+idx), inorder.slice(idx+1));
  return node;
};
\end{verbatim}

\subsection{参考题解,Golang}

Golang 的切片是不会拷贝数组的。

\begin{verbatim}
/**
 * Definition for a binary tree node.
 * type TreeNode struct {
 *     Val int
 *     Left *TreeNode
 *     Right *TreeNode
 * }
 */
func buildTree(preorder []int, inorder []int) *TreeNode {
  if len(preorder) == 0 || len(inorder) == 0 {
    return nil
  }
  val := preorder[0]
  idx := index(inorder, val)
  node := &TreeNode{Val: val}
  node.Left = buildTree(preorder[1:idx+1], inorder[:idx])
  node.Right = buildTree(preorder[idx+1:], inorder[idx+1:])
  return node
}

func index(array []int, val int) int {
  for i := 0; i < len(array); i++ {
    if array[i] == val {
      return i
    }
  }
  return -1
}
\end{verbatim}
