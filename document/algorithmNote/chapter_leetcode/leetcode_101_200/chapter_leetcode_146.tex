\newpage
\section{146. LRU缓存机制}
\label{leetcode:146}

\subsection{题目}

运用你所掌握的数据结构,设计和实现一个  LRU (最近最少使用) 缓存机制。
它应该支持以下操作: 获取数据 get 和 写入数据 put 。

获取数据 get(key) - 如果密钥 (key) 存在于缓存中,则获取密钥的值(总是正数),否则返回 -1。
写入数据 put(key, value) - 如果密钥不存在,则写入其数据值。当缓存容量达到上限时,
它应该在写入新数据之前删除最近最少使用的数据值,从而为新的数据值留出空间。

\textbf{进阶}:

你是否可以在 O(1) 时间复杂度内完成这两种操作?

\textbf{示例}:

\begin{verbatim}
  LRUCache cache = new LRUCache( 2 /* 缓存容量 */ );

  cache.put(1, 1);
  cache.put(2, 2);
  cache.get(1);       // 返回  1
  cache.put(3, 3);    // 该操作会使得密钥 2 作废
  cache.get(2);       // 返回 -1 (未找到)
  cache.put(4, 4);    // 该操作会使得密钥 1 作废
  cache.get(1);       // 返回 -1 (未找到)
  cache.get(3);       // 返回  3
  cache.get(4);       // 返回  4
\end{verbatim}

\subsection{题目分析}

这个题目就是实现一个 LRU cache,如果语言本身有自带的有序哈希表,
那就直接使用,如果没有的话,可以自己使用一个哈希表加上一个双向链表来实现。

\subsection{参考题解,JavaScript 代码}

\begin{verbatim}
/**
 * @param {number} capacity
 */
var LRUCache = function(capacity) {
  this.m = new Map();
  this.capacity = capacity;
};

/**
 * @param {number} key
 * @return {number}
 */
LRUCache.prototype.get = function(key) {
  if (!this.m.has(key)) {
    return -1;
  }
  const value = this.m.get(key);
  this.m.delete(key);
  this.m.set(key, value);
  return value;
};

/**
 * @param {number} key
 * @param {number} value
 * @return {void}
 */
LRUCache.prototype.put = function(key, value) {
  if (this.m.has(key)) {
    this.m.delete(key);
  } else {
    if (this.m.size + 1 > this.capacity) {
      this.m.delete(this.m.keys().next().value);
    }
  }
  this.m.set(key, value);
};

/**
 * Your LRUCache object will be instantiated and called as such:
 * var obj = new LRUCache(capacity)
 * var param_1 = obj.get(key)
 * obj.put(key,value)
 */
\end{verbatim}

\subsection{参考题解,Python 代码}

\begin{verbatim}
class LRUCache:
  def __init__(self, capacity: int):
    self.dict = collections.OrderedDict()
    self.remain = capacity

  def get(self, key: int) -> int:
    if key not in self.dict:
      return -1
    v = self.dict.pop(key)
    self.dict[key] = v
    return v

  def put(self, key: int, value: int) -> None:
    if key in self.dict:
      self.dict.pop(key)
    else:
      if self.remain > 0:
        self.remain -= 1
      else:
        self.dict.popitem(last=False)
    self.dict[key] = value

# Your LRUCache object will be instantiated and called as such:
# obj = LRUCache(capacity)
# param_1 = obj.get(key)
# obj.put(key,value)
\end{verbatim}
