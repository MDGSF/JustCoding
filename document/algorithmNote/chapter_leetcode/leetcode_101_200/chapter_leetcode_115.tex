\newpage
\section{115. 不同的子序列}
\label{leetcode:115}

\subsection{题目}

给定一个字符串 \textbf{S} 和一个字符串 \textbf{T},计算在 \textbf{S} 的子序列中 \textbf{T} 出现的个数。

一个字符串的一个子序列是指,通过删除一些(也可以不删除)字符且不干扰剩余
字符相对位置所组成的新字符串。(例如,``ACE'' 是 ``ABCDE'' 的一个子序列,而 ``AEC'' 不是)

\textbf{示例 1}:

\begin{verbatim}
  输入: S = "rabbbit", T = "rabbit"
  输出: 3
  解释:

  如下图所示, 有 3 种可以从 S 中得到 "rabbit" 的方案。
  (上箭头符号 ^ 表示选取的字母)

  rabbbit
  ^^^^ ^^
  rabbbit
  ^^ ^^^^
  rabbbit
  ^^^ ^^^
\end{verbatim}

\textbf{示例 2}:

\begin{verbatim}
  输入: S = "babgbag", T = "bag"
  输出: 5
  解释:

  如下图所示, 有 5 种可以从 S 中得到 "bag" 的方案。
  (上箭头符号 ^ 表示选取的字母)

  babgbag
  ^^ ^
  babgbag
  ^^    ^
  babgbag
  ^    ^^
  babgbag
    ^  ^^
  babgbag
      ^^^
\end{verbatim}

\subsection{参考题解,动态规划(递推)}

\begin{verbatim}
  dp[i][j] 表示 t 的前 i 个字符串可以由 s 的前 j 个字符串组成的最多个数
  s[j] == t[i], dp[i][j] = dp[i-1][j-1] + dp[i][j-1]
  s[j] != t[i], dp[i][j] = dp[i][j-1]
\end{verbatim}

\subsubsection{Python}

\begin{verbatim}
class Solution:
  def numDistinct(self, s: str, t: str) -> int:
    rows = len(t) + 1
    cols = len(s) + 1
    dp = [[0 for _ in range(cols)] for _ in range(rows)]
    for col in range(cols):
      dp[0][col] = 1
    for row in range(1, rows):
      for col in range(1, cols):
        if s[col-1] == t[row-1]:
          dp[row][col] = dp[row-1][col-1] + dp[row][col-1]
        else:
          dp[row][col] = dp[row][col-1]
    return dp[-1][-1]
\end{verbatim}

