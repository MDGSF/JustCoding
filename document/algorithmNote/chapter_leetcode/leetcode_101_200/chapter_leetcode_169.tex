\newpage
\section{169. 多数元素}
\label{leetcode:169}

\subsection{题目}

给定一个大小为 n 的数组,找到其中的多数元素。
多数元素是指在数组中出现次数\textbf{大于} ⌊ n/2 ⌋ 的元素。

你可以假设数组是非空的,并且给定的数组总是存在多数元素。

\textbf{示例 1}:

\begin{verbatim}
  输入: [3,2,3]
  输出: 3
\end{verbatim}

\textbf{示例 2}:

\begin{verbatim}
  输入: [2,2,1,1,1,2,2]
  输出: 2
\end{verbatim}

\subsection{参考题解,哈希表}

用哈希表把每个数字出现的次数都记录下来,
然后看下哪个数字出现的次数最多就是我们要的了。

\begin{verbatim}
/**
 * @param {number[]} nums
 * @return {number}
 */
var majorityElement = function(nums) {
  const m = {};
  for (let i = 0; i < nums.length; i += 1) {
    const num = nums[i];
    if (num in m) {
      m[num] += 1;
    } else {
      m[num] = 1;
    }
    if (m[num] > nums.length / 2) {
      return num;
    }
  }
};
\end{verbatim}

\subsection{参考题解,经典题解}

\begin{verbatim}
/**
 * @param {number[]} nums
 * @return {number}
 */
var majorityElement = function(nums) {
  let count = 0;
  let candidate = null;
  for (let i = 0; i < nums.length; i += 1) {
    const num = nums[i];
    if (count === 0) {
      candidate = num;
    }
    count += num === candidate ? 1 : -1;
  }
  return candidate;
};
\end{verbatim}
