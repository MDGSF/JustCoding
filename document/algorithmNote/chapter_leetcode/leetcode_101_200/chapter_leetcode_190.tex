\newpage
\section{190. 颠倒二进制位}
\label{leetcode:190}

\subsection{题目}

颠倒给定的 32 位无符号整数的二进制位。

\textbf{示例 1}:

\begin{verbatim}
输入: 00000010100101000001111010011100
输出: 00111001011110000010100101000000
解释: 输入的二进制串 00000010100101000001111010011100 表示无符号整数 43261596,
      因此返回 964176192,其二进制表示形式为 00111001011110000010100101000000。
\end{verbatim}

\textbf{示例 2}:

\begin{verbatim}
输入:11111111111111111111111111111101
输出:10111111111111111111111111111111
解释:输入的二进制串 11111111111111111111111111111101 表示无符号整数 4294967293,
      因此返回 3221225471 其二进制表示形式为 10101111110010110010011101101001。
\end{verbatim}

\subsection{参考题解}

\begin{verbatim}
def reverseBits(self, n: int) -> int:
  result = 0
  for i in range(32):
    result = result | (((n >> (31 - i)) & 1) << i)
  return result
\end{verbatim}
