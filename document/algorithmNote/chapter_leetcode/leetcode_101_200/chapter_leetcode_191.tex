\newpage
\section{191. 位1的个数}
\label{leetcode:191}

\subsection{题目}

编写一个函数,输入是一个无符号整数,返回其二进制表达式中数字位数为 ‘1’ 的个数(也被称为汉明重量)。

\textbf{示例 1}:

\begin{verbatim}
  输入:00000000000000000000000000001011
  输出:3
  解释:输入的二进制串 00000000000000000000000000001011 中,共有三位为 '1'。
\end{verbatim}

\textbf{示例 2}:

\begin{verbatim}
  输入:00000000000000000000000010000000
  输出:1
  解释:输入的二进制串 00000000000000000000000010000000 中,共有一位为 '1'。
\end{verbatim}

\textbf{示例 3}:

\begin{verbatim}
  输入:11111111111111111111111111111101
  输出:31
  解释:输入的二进制串 11111111111111111111111111111101 中,共有 31 位为 '1'。
\end{verbatim}

\subsection{参考题解1}

遍历所有的位,循环次数为 n 的位数。

\begin{verbatim}
def hammingWeight(self, n: int) -> int:
  count = 0
  while n != 0:
    count += n & 1
    n = n >> 1
  return count
\end{verbatim}

\subsection{参考题解2}

循环次数为 n 的二进制中 1 的个数。

\begin{verbatim}
def hammingWeight(self, n: int) -> int:
  count = 0
  while n > 0:
    n = n & (n - 1)
    count += 1
  return count
\end{verbatim}
