\newpage
\section{151. 翻转字符串里的单词}
\label{leetcode:151}

\subsection{题目}

给定一个字符串,逐个翻转字符串中的每个单词。

\textbf{示例 1}:

\begin{verbatim}
  输入: "the sky is blue"
  输出: "blue is sky the"
\end{verbatim}

\textbf{示例 2}:

\begin{verbatim}
  输入: "  hello world!  "
  输出: "world! hello"
  解释: 输入字符串可以在前面或者后面包含多余的空格,但是反转后的字符不能包括。
\end{verbatim}

\textbf{示例 3}:

\begin{verbatim}
  输入: "a good   example"
  输出: "example good a"
  解释: 如果两个单词间有多余的空格,将反转后单词间的空格减少到只含一个。
\end{verbatim}

\textbf{说明}:

\begin{itemize}
  \item 无空格字符构成一个单词。
  \item 输入字符串可以在前面或者后面包含多余的空格,但是反转后的字符不能包括。
  \item 如果两个单词间有多余的空格,将反转后单词间的空格减少到只含一个。
\end{itemize}

\textbf{进阶}:

请选用 C 语言的用户尝试使用 O(1) 额外空间复杂度的原地解法。

\subsection{参考题解}

\subsubsection{Python}

\begin{verbatim}
class Solution:
  def reverseWords(self, s: str) -> str:
    return ' '.join(list(reversed(s.strip().split())))
\end{verbatim}
