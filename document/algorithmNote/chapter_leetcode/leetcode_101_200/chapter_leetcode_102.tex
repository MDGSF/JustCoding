\newpage
\section{102. 二叉树的层次遍历}
\label{leetcode:102}

\subsection{题目}

给定一个二叉树,返回其按层次遍历的节点值。 (即逐层地,从左到右访问所有节点)。

例如:
给定二叉树: [3,9,20,null,null,15,7],

\begin{verbatim}
   3
  / \
 9  20
   /  \
  15   7
\end{verbatim}

返回其层次遍历结果:

\begin{verbatim}
[
  [3],
  [9,20],
  [15,7]
]
\end{verbatim}

\subsection{参考题解,BFS}

\begin{verbatim}
/**
 * Definition for a binary tree node.
 * function TreeNode(val) {
 *     this.val = val;
 *     this.left = this.right = null;
 * }
 */
/**
 * @param {TreeNode} root
 * @return {number[][]}
 */
var levelOrder = function(root) {
  if (root === null) { return []; }
  let result = [];
  let queue = [];
  queue.push(root);
  while (queue.length > 0) {
    const levelSize = queue.length;
    const currentLevel = [];
    for (let i = 0; i < levelSize; i += 1) {
      let node = queue.shift();
      currentLevel.push(node.val);
      if (node.left) { queue.push(node.left); }
      if (node.right) { queue.push(node.right); }
    }
    result.push(currentLevel);
  }
  return result;
};
\end{verbatim}

\subsection{参考题解,DFS}

\begin{verbatim}
/**
 * Definition for a binary tree node.
 * function TreeNode(val) {
 *     this.val = val;
 *     this.left = this.right = null;
 * }
 */
/**
 * @param {TreeNode} root
 * @return {number[][]}
 */
var levelOrder = function(root) {
  if (root === null) { return []; }
  let result = [];
  dfs(result, root, 0);
  return result;
};

function dfs(result, node, level) {
  if (node === null) { return; }
  if (result.length < level + 1) { result.push([]); }
  result[level].push(node.val);
  dfs(result, node.left, level + 1);
  dfs(result, node.right, level + 1);
}
\end{verbatim}
