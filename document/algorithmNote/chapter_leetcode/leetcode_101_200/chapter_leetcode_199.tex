\newpage
\section{199. 二叉树的右视图}
\label{leetcode:199}

\subsection{题目}

给定一棵二叉树,想象自己站在它的右侧,按照从顶部到底部的顺序,返回从右侧所能看到的节点值。

\textbf{示例}:

\begin{verbatim}
  输入: [1,2,3,null,5,null,4]
  输出: [1, 3, 4]
  解释:

    1            <---
  /   \
 2     3         <---
  \     \
   5     4       <---
\end{verbatim}

\subsection{参考题解,DFS}

先遍历右儿子,再遍历左儿子,
可以参考 \hyperref[leetcode:102]{102. 二叉树的层次遍历} 的 DFS 解法。

\subsubsection{Python}

\begin{verbatim}
# Definition for a binary tree node.
# class TreeNode:
#     def __init__(self, x):
#         self.val = x
#         self.left = None
#         self.right = None

class Solution:
  def rightSideView(self, root: TreeNode) -> List[int]:
    result = []
    def dfs(root, level):
      if root == None: return
      if level + 1 > len(result):
        result.append(root.val)
      dfs(root.right, level + 1)
      dfs(root.left, level + 1)
    dfs(root, 0)
    return result
\end{verbatim}
