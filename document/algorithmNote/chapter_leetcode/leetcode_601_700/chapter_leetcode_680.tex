\newpage
\section{680. 验证回文字符串 Ⅱ}
\label{leetcode:680}

\subsection{题目}

给定一个非空字符串 s,\textbf{最多}删除一个字符。判断是否能成为回文字符串。

\textbf{示例 1}:

\begin{verbatim}
  输入: "aba"
  输出: True
\end{verbatim}

\textbf{示例 2}:

\begin{verbatim}
  输入: "abca"
  输出: True
  解释: 你可以删除c字符。
\end{verbatim}

\textbf{注意}:

字符串只包含从 a-z 的小写字母。字符串的最大长度是50000。

\subsection{参考题解,傻递归}

\subsubsection{Python}

\begin{verbatim}
class Solution:
  def validPalindrome(self, s: str) -> bool:
    def dfs(left, right, canRemove):
      if left >= right:
        return True
      if s[left] == s[right]:
        return dfs(left + 1, right - 1, canRemove)
      elif canRemove:
        return dfs(left + 1, right, False) or dfs(left, right - 1, False)
      else:
        return False
    return dfs(0, len(s) - 1, True)
\end{verbatim}

\subsection{参考题解}

\subsubsection{Python}

\begin{verbatim}
class Solution:
  def validPalindrome(self, s: str) -> bool:
    if s == s[::-1]: return True
    left, right = 0, len(s) - 1
    while left < right:
      if s[left] != s[right]:
        return s[left:right] == s[left:right][::-1] or \
          s[left + 1:right + 1] == s[left + 1:right + 1][::-1]
      left, right = left + 1, right - 1
    return True
\end{verbatim}
