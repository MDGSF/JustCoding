\newpage
\section{641. 设计循环双端队列}
\label{leetcode:641}

\subsection{题目}

设计实现双端队列。\\
你的实现需要支持以下操作:

\begin{enumerate}
  \item MyCircularDeque(k):构造函数,双端队列的大小为k。
  \item insertFront():将一个元素添加到双端队列头部。 如果操作成功返回 true。
  \item insertLast():将一个元素添加到双端队列尾部。如果操作成功返回 true。
  \item deleteFront():从双端队列头部删除一个元素。 如果操作成功返回 true。
  \item deleteLast():从双端队列尾部删除一个元素。如果操作成功返回 true。
  \item getFront():从双端队列头部获得一个元素。如果双端队列为空,返回 -1。
  \item getRear():获得双端队列的最后一个元素。 如果双端队列为空,返回 -1。
  \item isEmpty():检查双端队列是否为空。
  \item isFull():检查双端队列是否满了。
\end{enumerate}

\textbf{示例}:

\begin{verbatim}
  MyCircularDeque circularDeque = new MycircularDeque(3); // 设置容量大小为3
  circularDeque.insertLast(1);			        // 返回 true
  circularDeque.insertLast(2);			        // 返回 true
  circularDeque.insertFront(3);			        // 返回 true
  circularDeque.insertFront(4);			        // 已经满了,返回 false
  circularDeque.getRear();  				// 返回 2
  circularDeque.isFull();				        // 返回 true
  circularDeque.deleteLast();			        // 返回 true
  circularDeque.insertFront(4);			        // 返回 true
  circularDeque.getFront();				// 返回 4
\end{verbatim}

\textbf{提示}:

\begin{enumerate}
  \item 所有值的范围为 [1, 1000]
  \item 操作次数的范围为 [1, 1000]
  \item 请不要使用内置的双端队列库。
\end{enumerate}

\subsection{参考题解}

\begin{verbatim}
/**
 * Initialize your data structure here. 
 * Set the size of the deque to be k.
 * @param {number} k
 */
var MyCircularDeque = function(k) {
  this.array = new Array(k);
  this.head = 0;
  this.rear = 0;
  this.len = 0;
  this.capacity = k;
};

/**
 * Adds an item at the front of Deque. 
 * Return true if the operation is successful.
 * @param {number} value
 * @return {boolean}
 */
MyCircularDeque.prototype.insertFront = function(value) {
  if (this.isFull()) {
    return false;
  }
  this.array[this.head] = value;
  this.head = (this.head + 1) % this.capacity;
  this.len += 1;
  return true;
};

/**
 * Adds an item at the rear of Deque. 
 * Return true if the operation is successful.
 * @param {number} value
 * @return {boolean}
 */
MyCircularDeque.prototype.insertLast = function(value) {
  if (this.isFull()) {
    return false;
  }
  if (this.rear === 0) {
    this.rear = this.capacity - 1;
  } else {
    this.rear -= 1;
  }
  this.array[this.rear] = value;
  this.len += 1;
  return true;
};

/**
 * Deletes an item from the front of Deque. 
 * Return true if the operation is successful.
 * @return {boolean}
 */
MyCircularDeque.prototype.deleteFront = function() {
  if (this.isEmpty()) {
    return false;
  }
  if (this.head === 0) {
    this.head = this.capacity - 1;
  } else {
    this.head -= 1;
  }
  this.len -= 1;
  return true;
};

/**
 * Deletes an item from the rear of Deque. 
 * Return true if the operation is successful.
 * @return {boolean}
 */
MyCircularDeque.prototype.deleteLast = function() {
  if (this.isEmpty()) {
    return false;
  }
  this.rear = (this.rear + 1) % this.capacity;
  this.len -= 1;
  return true;
};

/**
 * Get the front item from the deque.
 * @return {number}
 */
MyCircularDeque.prototype.getFront = function() {
  if (this.isEmpty()) {
    return -1;
  }
  let prev = this.head === 0 ? this.capacity - 1 : this.head - 1;
  return this.array[prev];
};

/**
 * Get the last item from the deque.
 * @return {number}
 */
MyCircularDeque.prototype.getRear = function() {
  if (this.isEmpty()) {
    return -1;
  }
  return this.array[this.rear];
};

/**
 * Checks whether the circular deque is empty or not.
 * @return {boolean}
 */
MyCircularDeque.prototype.isEmpty = function() {
  return this.len === 0;
};

/**
 * Checks whether the circular deque is full or not.
 * @return {boolean}
 */
MyCircularDeque.prototype.isFull = function() {
  return this.len === this.capacity;
};

/**
 * Your MyCircularDeque object will be instantiated and called as such:
 * var obj = new MyCircularDeque(k)
 * var param_1 = obj.insertFront(value)
 * var param_2 = obj.insertLast(value)
 * var param_3 = obj.deleteFront()
 * var param_4 = obj.deleteLast()
 * var param_5 = obj.getFront()
 * var param_6 = obj.getRear()
 * var param_7 = obj.isEmpty()
 * var param_8 = obj.isFull()
 */
\end{verbatim}
