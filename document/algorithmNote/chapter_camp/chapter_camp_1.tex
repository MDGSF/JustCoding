\subsection{第 1 课}

\subsubsection{脑图}

\href{https://github.com/MDGSF/algorithm005-class02/blob/master/Week_00/G20190343020039/%E7%AE%97%E6%B3%95%E8%84%91%E5%9B%BE.jpeg}{数据结构与算法脑图}

\subsubsection{学习笔记}

\paragraph{五毒神掌}

\begin{itemize}
  \item 自己先思考 5 分钟,有想法自己先做一遍。没有想法就直接看题解,看懂之后做一遍。
  \item 没有想法看题解的话,要自己不看题解再做一遍。
  \item 第二天再做一遍。
  \item 一个星期之后再做一遍。
  \item 面试的前一个星期再做一遍。
\end{itemize}

``五毒神掌''的核心思想在于过遍数,你一天刷个几百遍是没有用的,
要每隔一段时间就刷一遍。因为这是符合人类的遗忘曲线的。

\paragraph{刷题最大误区}

\begin{itemize}
  \item 刷题只刷一遍。刷一遍是远远不够的。
  \item 我要自己想出题目的解法。很多题目你要是能自己想出来的
    话,你都可以拿图灵奖了,醒醒吧少年。
\end{itemize}

\paragraph{切题四件套}

\begin{itemize}
  \item Clarification,确保自己对题目的理解是正确的。
  \item Possible Solutions,对比不同解法的 (time/space)。
  \item Coding,编码。
  \item Test cases,编写测试用例。
\end{itemize}
