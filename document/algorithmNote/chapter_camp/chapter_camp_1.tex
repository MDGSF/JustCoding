\newpage
\section{第 1 课}

\subsection{脑图}

\href{https://github.com/MDGSF/algorithm005-class02/blob/master/Week_00/G20190343020039/%E7%AE%97%E6%B3%95%E8%84%91%E5%9B%BE.jpeg}{数据结构与算法脑图}

\subsection{学习笔记}

其实预习周里,覃超老师教的东西才是最有用的。 
``五毒神掌''这样的方法可以说是颠覆了自己的认知。 
自己以前一直都是想要自己去想题目的解决方法, 
结果可想而知,自然是碰到了很多的困难和挫折。 
毕竟你要是能想出那么精妙的算法的话,你都能拿图灵奖了。
所以学算法的第一步就是要先清楚的给自己划定一条边界:
你是要学习算法而不是要去 发明创造一个新的算法。
其实也很容易理解,你要是没有学习过加减乘除,
让你自己去发明加减乘除,试问又有几个人能够办到。
我们更多的时候,是站在前人的基础上。牛顿曾说过:
\begin{quote}
``我之所以能 成功,是因为我站在巨人的肩膀上''
\end{quote}
所以对于我们普通人而言,你更多的是需要认真学习。
