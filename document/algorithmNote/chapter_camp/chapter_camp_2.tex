\subsection{第 2 课 | 训练准备和复杂度分析}

\subsubsection{开发环境}

\begin{quote}
工欲善其事,必先利其器
\end{quote}

每个人最好都有一个自己擅长的编辑器,这会让你的开发事半功倍。
我自己是使用\textbf{Vim}和\textbf{VSCode}。你要是不知道
使用哪个好,那就用\textbf{VSCode}准没错。

那怎么学习呢?直接在 Google 搜索 Top tips for VSCode

\subsubsection{时间复杂度}

\begin{itemize}
  \item O(1),Constant Complexity 常数复杂度,最佳,比如 Hash 表,缓存等。
  \item O($log_2 n$),Logarithmic Complexity 对数复杂度,
    仅次于常数复杂度,如二分查找、二叉搜索树等。
  \item O(n),Linear Complexity 线性复杂度,如大多数遍历操作。
  \item O(nlogn) 快速排序时间复杂度
  \item O(n$^{2}$) N square Complexity 平方,双重 for 循环
  \item O(n$^{3}$) N cubic Complexity 立方, 3 重 for 循环
  \item O(2$^{n}$),O(3$^{n}$),O(k$^{n}$) Exponential Growth 指数,
    递归的时间复杂度
  \item O(n!) 阶乘时间复杂度
\end{itemize}
