\newpage
\section{毕业总结}

一眨眼间算法训练营就已经结束了。这3个月的算法训练让我获益良多,我在这里真心的
感谢覃超老师。我是在极客时间的``算法面试通关40讲''认识覃超老师的。之前,我一直
都有尝试过再补补算法方面的知识,但是却屡屡受挫,不是被``算法导论'' 摁在地板上
摩擦,就是被传统OJ欺负的鼻青脸肿。直到看了超哥的``算法面试通关40讲''之后,我发
现超哥对题目的讲解简单易懂,而且最重要的是超哥还教你自己学习的方法,而不仅仅只
是把题目讲讲了事。

在算法训练营期间,因为工作原因,每天都只能晚上10半之后才开始刷题,好在是周末没有
上班。所以我主要的学习时间是放在周末。整个学习过程还是比较充实的。

我就在想,我要是大学那会就能碰到想超哥这样的领路人就好了。不过也没事,种树的最佳
时间是十年前,其次就是当下。算法训练营的结束并不意味着算法学习结束了,而是一个
新的开始。只要能坚持下去,算法这块的内容一定不会成为自己的短板。我在leetcode上面
的题目数量已经有300+了,在接下来的一年里,准备先把超哥教的全部重新过一遍,然后
争取刷到500+的题目。

我这里把最重要的东西总结如下:

\paragraph{第一、五遍刷题法}

\begin{itemize}
  \item 看完题目,到国际站看题解
  \item 不看题解,自己做一遍
  \item 24小时之后再做一遍
  \item 一周之后再做一遍
  \item 面试前一周再做一遍
\end{itemize}

\paragraph{第二、刷题的最大误区} \

只做一遍题目。

\paragraph{第三、到哪里刷题?} \

在leetcode中国站刷题,到leetcode国际站看题解。

\paragraph{第四、不要死磕一道题目} \

算法本身就已经很难了,先要建立自己学习的信心。看到一道题目,10分钟没有思路就直接
看题解。

